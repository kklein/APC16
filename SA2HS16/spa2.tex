%%%%%%%%%%%%%%%%%%%%%%%%%%%%%%%%%%%%%%%%%
% Programming/Coding Assignment
% LaTeX Template
%
% Original author:
% Ted Pavlic (http://www.tedpavlic.com)
%
%
% This template uses a Perl script as an example snippet of code, most other
% languages are also usable. Configure them in the "CODE INCLUSION 
% CONFIGURATION" section.
%
%%%%%%%%%%%%%%%%%%%%%%%%%%%%%%%%%%%%%%%%%

%----------------------------------------------------------------------------------------
%	PACKAGES AND OTHER DOCUMENT CONFIGURATIONS
%----------------------------------------------------------------------------------------

\documentclass{article}
\usepackage[utf8]{inputenc}

\usepackage[german]{babel}
\usepackage{amsmath}
\usepackage{amsfonts}
\usepackage{fancyhdr} % Required for custom headers
\usepackage{lastpage} % Required to determine the last page for the footer
\usepackage{extramarks} % Required for headers and footers
\usepackage[usenames,dvipsnames]{color} % Required for custom colours
\usepackage{graphicx} % Required to insert images
\usepackage{listings} % Required for insertion of code
\usepackage{courier} % Required for the courier font
\usepackage{enumerate} % used for enumerate args
\usepackage{multicol} % columns

\usepackage{pgf} 
\usepackage{tikz}
%\usepackage{forest} % treees :D
%\usetikzlibrary{arrows,automata} %for FSM

% Custom commands
\DeclareMathOperator{\Kl}{Kl} %Klassen von Zuständen

\usepackage{mathtools}
\DeclarePairedDelimiter{\ceil}{\lceil}{\rceil}
% Shamelessly copied from http://tex.stackexchange.com/questions/43008/absolute-value-symbols
\DeclarePairedDelimiter\abs{\lvert}{\rvert} % nice |x|
\DeclarePairedDelimiter\norm{\lVert}{\rVert} % nice ||x||
% Swap the definition of \abs* and \norm*, so that \abs
% and \norm resizes the size of the brackets, and the 
% starred version does not.
\makeatletter
\let\oldabs\abs
\def\abs{\@ifstar{\oldabs}{\oldabs*}}
\let\oldnorm\norm
\def\norm{\@ifstar{\oldnorm}{\oldnorm*}}
\makeatother


% Margins
\topmargin=-0.45in
\evensidemargin=0in
\oddsidemargin=0in
\textwidth=6.5in
\textheight=9.0in
\headsep=0.25in

\linespread{1.1} % Line spacing

% Set up the header and footer
\pagestyle{fancy}
\lhead{\hmwkAuthorName} % Top left header
%\chead{\hmwkClass\ (\hmwkClassInstructor\): \hmwkTitle} % Top center head
%\rhead{\firstxmark} % Top right header
\rhead{}
\lfoot{\lastxmark} % Bottom left footer
\cfoot{} % Bottom center footer
\rfoot{page\ \thepage\ of\ \protect\pageref{LastPage}} % Bottom right footer
\renewcommand\headrulewidth{0.4pt} % Size of the header rule
\renewcommand\footrulewidth{0.4pt} % Size of the footer rule

\setlength\parindent{0pt} % Removes all indentation from paragraphs

%----------------------------------------------------------------------------------------
%	CODE INCLUSION CONFIGURATION
%----------------------------------------------------------------------------------------

%\definecolour{MyDarkGreen}{rgb}{0.0,0.4,0.0} % This is the colour used for comments
%\lstloadlanguages{Pascal} % Load Pascal syntax for listings, for a list of other languages supported see: ftp://ftp.tex.ac.uk/tex-archive/macros/latex/%contrib/listings/listings.pdf
%\lstset{language=Perl, % Use Pascal in this example
%        frame=single, % Single frame around code
%        basicstyle=\small\ttfamily, % Use small true type font
%        keywordstyle=[1]\colour{Blue}\bf, % Pascal functions bold and blue
%        keywordstyle=[2]\colour{Purple}, % Pascal function arguments purple
%        keywordstyle=[3]\colour{Blue}\underbar, % Custom functions underlined and blue
%        identifierstyle=, % Nothing special about identifiers                                         
%        commentstyle=\usefont{T1}{pcr}{m}{sl}\colour{MyDarkGreen}\small, % Comments small dark green courier font
%        stringstyle=\colour{Purple}, % Strings are purple
%        showstringspaces=false, % Don't put marks in string spaces
%        tabsize=5, % 5 spaces per tab
%        %
%        % Put standard Pascal functions not included in the default language here
%        morekeywords={rand},
%        %
%        % Put Pascal function parameters here
%        morekeywords=[2]{on, off, interp},
%        %
%        % Put user defined functions here
%        morekeywords=[3]{test},
%        %
%        morecomment=[l][\colour{Blue}]{...}, % Line continuation (...) like blue comment
%        numbers=left, % Line numbers on left
%        firstnumber=1, % Line numbers start with line 1
%        numberstyle=\tiny\colour{Blue}, % Line numbers are blue and small
%        stepnumber=5 % Line numbers go in steps of 5
%}

\definecolor{dkgreen}{rgb}{0,0.6,0}
\definecolor{gray}{rgb}{0.5,0.5,0.5}
\definecolor{mauve}{rgb}{0.58,0,0.82}

\lstset{frame=tb,
  language=Java,
  aboveskip=3mm,
  belowskip=3mm,
  showstringspaces=false,
  columns=flexible,
  basicstyle={\small\ttfamily},
  numbers=none,
  numberstyle=\tiny\colour{gray},
  keywordstyle=\colour{blue},
  commentstyle=\colour{dkgreen},
  stringstyle=\colour{mauve},
  breaklines=true,
  breakatwhitespace=true,
  tabsize=3
}

% Creates a new command to include a perl script, the first parameter is the filename of the script (without .p), the second parameter is the caption
\newcommand{\pascalscript}[2]{
\begin{itemize}
\item[]\lstinputlisting[caption=#2,label=#1]{#1.p}
\end{itemize}
}

%----------------------------------------------------------------------------------------
%	DOCUMENT STRUCTURE COMMANDS
%	Skip this unless you know what you're doing
%----------------------------------------------------------------------------------------

% Header and footer for when a page split occurs within a problem environment
%\newcommand{\enterProblemHeader}[1]{
%\nobreak\extramarks{#1}{#1 continued on next page\ldots}\nobreak
%\nobreak\extramarks{#1 (continued)}{#1 continued on next page\ldots}\nobreak
%}

% Header and footer for when a page split occurs between problem environments
%\newcommand{\exitProblemHeader}[1]{
%\nobreak\extramarks{#1 (continued)}{#1 continued on next page\ldots}\nobreak
%\nobreak\extramarks{#1}{}\nobreak
%}

\setcounter{secnumdepth}{0} % Removes default section numbers
\newcounter{homeworkProblemCounter} % Creates a counter to keep track of the number of problems

\newcommand{\homeworkProblemName}{}
\newenvironment{homeworkProblem}[1][Exercise \arabic{homeworkProblemCounter}]{ % Makes a new environment called homeworkProblem which takes 1 argument (custom name) but the default is "Problem #"
\stepcounter{homeworkProblemCounter} % Increase counter for number of problems
\renewcommand{\homeworkProblemName}{#1} % Assign \homeworkProblemName the name of the problem
\section{\homeworkProblemName} % Make a section in the document with the custom problem count
%\enterProblemHeader{\homeworkProblemName} % Header and footer within the environment
}{
%\exitProblemHeader{\homeworkProblemName} % Header and footer after the environment
}

\newcommand{\problemAnswer}[1]{ % Defines the problem answer command with the content as the only argument
\noindent\framebox[\columnwidth][c]{\begin{minipage}{0.98\columnwidth}#1\end{minipage}} % Makes the box around the problem answer and puts the content inside
}

\newcommand{\homeworkSectionName}{}
\newenvironment{homeworkSection}[1]{ % New environment for sections within homework problems, takes 1 argument - the name of the section
\renewcommand{\homeworkSectionName}{#1} % Assign \homeworkSectionName to the name of the section from the environment argument
\subsection{\homeworkSectionName} % Make a subsection with the custom name of the subsection
%\enterProblemHeader{\homeworkProblemName\ [\homeworkSectionName]} % Header and footer within the environment
}{
%\enterProblemHeader{\homeworkProblemName} % Header and footer after the environment
}

%----------------------------------------------------------------------------------------
%	NAME AND CLASS SECTION
%----------------------------------------------------------------------------------------

\newcommand{\hmwkTitle}{Special Assignment 2} % Assignment title
\newcommand{\hmwkDueDate}{6\ December\ 2016} % Due date
\newcommand{\hmwkClass}{Algorithms, Probability and Computing} % Course/class
 \newcommand{\hmwkClassInstructor}{} % Teacher/lecturer
\newcommand{\hmwkAuthorName}{Kevin Klein} % Your name

%----------------------------------------------------------------------------------------
%	TITLE PAGE
%----------------------------------------------------------------------------------------

\title{
\vspace{2in}
\textmd{\textbf{\hmwkClass:\ \hmwkTitle}}\\
\normalsize\vspace{0.1in}\small{Due\ \hmwkDueDate}\\
\vspace{0.1in}\large{\textit{\hmwkClassInstructor}
\vspace{3in}
}}
\author{\textbf{\hmwkAuthorName}}
\date{} % Insert date here if you want it to appear below your name

%----------------------------------------------------------------------------------------

\begin{document}

\maketitle

%----------------------------------------------------------------------------------------
%	TABLE OF CONTENTS
%----------------------------------------------------------------------------------------

%\setcounter{tocdepth}{1} % Uncomment this line if you don't want subsections listed in the ToC

\addtocounter{homeworkProblemCounter}{0}
\newpage
%\tableofcontents
%\newpage

%----------------------------------------------------------------------------------------
%	PROBLEM 1
%----------------------------------------------------------------------------------------

\begin{homeworkProblem}
\begin{enumerate}[(a)]
%%%%% a)
\item
	For the sake of contradiction, let's assume \(\tau(F) < \mu(F)\).\\
	Say \( \tau(F) = min\ |T| = k\). Sizes of sets must be integers, hence \(k \in \mathbb{N}\). It follows directly that \(\mu(F)=max\ |M| \geq k+1\). The latter expresses that \(k+1\) flags can be simultaneously  represented without assigning a dancer to more than one flag. These \(k+1\) flags of the Olympic matching are by definition pairwise dstinct, meaning that no colour is present in more than one of those flags. \\
	By definition, an Olympic traversal has to pass all flags and thereby also the \(k+1\) flags in the Olympic matching. However, as those flags are pairwise distinct, the necessary amount of colours to traverse the matching is \(k+1\). This number can only increase when adding colours to traverse flags which are not part of the Olympic matching. Hence \(\tau(F) \geq k+1 \). \\
	This contradicts are assumption. Therefore we have proven that \( \mu(F) \leq \tau(F) \). 

%%%%% b)
\item
	Firstly, we need to show that such a \( \tau^{*}(F)\) exists. In order to do so we demonstrate that LP-T is bounded and feasible, which implies existance of an optimal solution. \\
	The cost function increases with every component of x. This is to say that it will attain minimal value if the components of x are 'as small as possible'. However, due to the constraint that \(x \geq 0 \), we know that the components of \(x\) must be at least \(0\). Thus, we can formulate the lower bound \( 1_n^T 0_n = 0 \) for the minimized result. It follows that LP-T is bounded. \\
	In order to show that LP-T is feasible, we will look at the structure of \(A\). The matrix solely consists of \(0's\) and \(1's\). Using the information that each flag comprises at least one colour, we can tell that every row holds at least one \(1\). In order to satisfy the constraint that the dotproduct of each row with \(x\) is larger or equal 1, we simply set all components of \(x\) to \(1\). Furthermore, \(x = 1_n\) also trivially satisfies the remaining constraint, namely \(x \leq 0\). Thererfore \(x\) witnesses the feasibility of LP-T. \\
	We now know that LP-T is bounded, feasible and thereby has an optimal solution \( \tau^{*}(F) \). \\
	When closeely observing the definition of the traversal \( T\) and \(\tau(F)\), it becomes apparent, that we can express it as an integer linear program with strong resemblance to LP-T. Instead of declaring \(T\) as a set, we can express \(T\) equivalently as the vector \(x\), indicating \(x_i = \mathbb{I}[c_i \in T ]\). It follows directly that \( \tau(F) = |T| = \sum_{i=1}^{n} x_i = 1_n ^Tx \). The constraint that the traversal \(x\) must include at least one colour of each flag, can be enforced by counting the number of colours each flag and the traversal have in common and making sure this count is always at least 1. This corresponds to the dotproduct of a row of \(A\), representing the colours in a flag, and the traversal \(x\). We can thereby formulate \(\tau(F)\) as the optimal value resulting from:
	\[ minimize\ 1_n^Tx,\ subject\ to\ Ax \geq 1_m,\ x \in \{0,1\}^n 
	\]
	We can argue analogously that this integer linear program has an optimal value. \\
	The integer LP we came to formulate is only more restrictive than LP-T. Hence, a traversal \(x\) leading to the optimal value for the integer linear program, will yield the same value for LP-T. Yet, LP-T might allow for an even better solution, as we will show in c). We have proven that \(\tau^{*}(F) \leq \tau(F)\).

%%%%% c)
\item
	Consider the flags \( f_1 = \{c1,c2\},\ f_2 = \{c1,c3\},\ f_3 = \{c2,c3\} \). \\
	In other words:
	\[
	A = \begin{bmatrix}
		1 & 1 & 0 \\
		1 & 0 & 1 \\
		0 & 1 & 1 \\
	\end{bmatrix}
	\]
	We see that no colour is contained in all three flags. Hence the traversal will require at least two colours. We write \( \tau(F) \geq 2 \). 
	Let us inspect \( x_s = \begin{bmatrix}
		0.5 & 0.5 & 0.5 \\
	\end{bmatrix}^T \). Firstly, our \(x_s\) satisfies \(x \geq 0_n\). Secondly 
	\[
	A \cdot x_s =  \begin{bmatrix}
		1 & 1 & 0 \\
		1 & 0 & 1 \\
		0 & 1 & 1 \\
	\end{bmatrix} \cdot \begin{bmatrix}
		0.5 \\
		0.5 \\
		0.5 \\
	\end{bmatrix} =  \begin{bmatrix}
		1 \\
		1 \\
		1 \\
	\end{bmatrix} \geq 1_m.
	\]
	Thereby, we can conclude that \(x_s\) is a feasible solution for LP-T. Furthermore, the optimal value \(\tau^{*} (F) \) must be smaller or equal to the value that the cost function yields for $x_s$. This follows from the simple fact that the cost function is minimizing. Hence \(\tau^{*}(F) \leq 1_n^T \cdot x_s = 1.5\). \\
	It follows directly that \(\tau^{*}(F) < \tau(F) \).
	
%%%% d)
\item
	In order to prove \(\mu(F) \leq \tau^*(F) \) we will first show \( \mu(F) \leq \mu^{*}(F)\) followed by \( \mu^{*}(F) \leq \tau^*(F) \). \\
	We define \( \mu^{*}(F) \) to be the optimal value of the following linear program LP-D: \\
	\[ maximize\ 1_m^T y\ subject\ to\ A^Ty \leq 1,\ y \geq 0
	\]
	We observe that LP-D corresponds to the definition of the dual of LP-T. As we have shown that LP-T is bounded and has optimal value \(\tau^{*}(F)\), the Strong Duality Theorem tells us that LP-D has optimal value \(\mu^{*}(F) = \tau^{*}(F)\). \\
	Similarly as in (b), we will translate the meaning of an Olympic matching \(M\) to an integer linear program and show that this integer linear program is only more constrained than LP-D. \\
	Instead of declaring \(M\) as a set, we can express \(M\) as the vector \(y\), indicating \(y_i = \mathbb{I}[f_i \in M]\). It follows directly that \( \mu(F) = |M| = \sum_{i=1}^{m} y_i = 1_m ^Ty \). The constraint that the matching \(y\) must assign each colour to at most one flag, i.e. the flags are pairwise distinct, can be enforced by counting the number of flags each colour appears in and making sure this count is always at most 1. This corresponds to the dotproduct between a row of \(A^T\), representing for all flags whether they contain a specific colour, and the matching \(y\). We can thereby formulate \(\mu(F)\) as the optimal value resulting from:
	\[ maximize\ 1_m^T y\ subject\ to\ A^Ty \leq 1,\ y \in \{0,1\}^m
	\]
	\(y = 1_m\) yields an upper lower bound for \(\mu(F)\), namely \(1_m^{T}1_m = m\). \(y = 0_m\) yields an obvious feasible solution. Hence the integer linear program has an optimal solution \(\mu(F)\). \\
	The integer LP we came to formulate is only more restrictive than LP-D. Hence, a matching \(y\) leading to the optimal value for the integer linear program, will yield the same value for LP-D. Yet, LP-D might allow for an even better, i.e. larger, solution, as it is less restrictive and maximizing the cost function. We have proven that \(\mu(F) \leq \mu^{*}(F)\).
	
	We have proven that \(\mu(F) \leq \tau^*(F) \).

%%%% e)
\item
	Let's inspect the probablity to intersect a specific flag when drawing a colour according to \(p\). We intersecting if we fraw any of the colours containted in the flag. Using \(p\), this gives us:
	\[ \Pr [f_i] = \frac{\sum_{j=1}^n \mathbb{I}[c_j \in f_k]\cdot x_i^*} {\sum_{j=1}^n x_j^*} = \frac{A_i x^*}{\sum_{j=1}^n x_j^*} = \frac{A_i x^*}{\tau^*(F)}\]
	The probability of not intersecting a certain flag upon drawing a colour is hence:
	\[1 - \Pr [f_i] = \frac{\tau^*(F)}{\tau^*(F)} - \frac{A_i x^*}{\tau^*(F)} = \frac{1_n^Tx^*-A_i x^*}{\tau^*(F)} = \frac{(1_n^T-A_i) x^*}{\tau^*(F)}\] 
	The probability of not intersecting a certain flag upon drawing s colours, possibly with repetitions, is hence:
	\[(1 - \Pr [f_i])^s = (\frac{(1_n^T-A_i) x^*}{\tau^*(F)})^s\]
	After such handy clarifications, let's look at what we actually want to show.
	\begin{align}
		 \mathbb{E}[\# flags\ not\ intersected\ by\ s\ draws] &= \mathbb{E}[\sum_{i=1}^m \mathbb{I}[f_i\ not\ intersected]] \\
	&= \sum_{i=1}^m \mathbb{E}[f_i\ not\ intersected]\ \ \ \ (linearity\ of\ expectation) \\
	&= \sum_{i=1}^m \Pr[f_i\ not\ intersected] \\
	&= \sum_{i=1}^m (\frac{(1_n^T-A_i) x^*}{\tau^*(F)})^s \\
	&= \frac{1}{(\tau^*(F))^s} \cdot \sum_{i=1}^m ( (1_n^T-A_i) x^*)^s \\
	\end{align}
%%%% f)
\item
	\begin{enumerate}[(i)]
	%%%%% i)
	\item
		We want to show that the columns of \(A\) are linearly independent. \\
		Thanks to the special structure of \(A\), we know that each row contains exactly two \(1\)'s and all other components equal \(0\). \\
		If \(m < n\), we know that \( rank(A) \leq dim(rowspace(A)) \leq m\).
		We also know that \( number\ lin.\ ind.\ colums = rank(A) \). Hence \(number\ lin.\ ind.\ colums \leq m < n\). Therefore the columns are linearly dependent. \\
		If \(m \geq n \), we proceed by contradiction. Let's assume that \(A\) has n linear independent columns and hence \(rank(A) = n\). The 'rank-nullity theorem' tells us that \( n = rank(A) + dim(kernel(A)) \). Therefore the dimension of the kernel is 0 and \(Av = 0\) can only hold for \(v = 0\). Say there are \(k \) light colours and hence \(n-k\) dark colours. We construct the vector:
		\[v_0 \in \{0,1\}^n,\ v_0[i] = \begin{cases}1\ & if\ c_i\ bright\\ -1\ & otherwise\end{cases}\]
		As we know that each row of \(A\) contains a 1 for exactly one bright and one dark colour and all other elements 0, all of the multiplications between rows and \(v_0\) will return 0. In other words, \(v \neq 0, Av = 0\). Therefore the kernel of A must have at least dimension 1. This contradicts our assumption that A has rank n. Thereby, A cannot have n linearly independent columns. 
	%%%%% ii)
	\item
	In order to show that for each square submatrix \(S\) of \(A\) \(det|S| \in \{-1,0,1\} \), we will proceed by induction. We will say that \(S\) is of size \(i\) by \(i\). \\
	\textit{Base case: i = 1} \\
	\[\forall i \in \{1,..,m\},\ \forall j \in \{1,..,n\}\ A[i,j] \in \{0,1\}\ \Rightarrow det|S| \in \{0,1\}\]
	\textit{Induction step: } \(i \rightarrow i+1\) \\
	In order to make use of our induction hypothesis, we want to compute \(det|S|\), with \(S\) of size \(i+1\) by \(i+1\), by its minors, of size \(i\) by \(i\). Thanks to the fact that \(S\) is a submatrix of \(A\), we can assume that each row of \(S\) has either no, one or two \(1\)'s. Furthermore we know that we can pick any row to calculate \(det|S|\) by corresponding minors. 
	We make the following case distinction:
	\begin{enumerate}[a)]
		\item There is one row in \(S\) with all \(0\)'s. \\
			\(det|S|\) will be calculated by adding the determinant of all minors, scaled by the corrsponding elements of the row. As the latter are all \(0\), it follows immediately that \(det|S| = 0\). 
		\item There is no row in row in \(S\) with all \(0\)'s, there is one row in \(S\) with exactly one \(1\). \\
			\(det|S|\) will be calculated by adding the determinant of all minors, scaled by the corrsponding elements of the row. Hence all determinants of minors but one, call it \(M\) can be ignored. Our induction hypotehsis tells us that \(det|M| \in \{-1,0,1\} \) because \(M\) is of size \(i\) by \(i\) and by definition a submatrix of \(A\). Hence \(det|S|=1\cdot det|M|\) which direcly implies our claim.
		\item There are only rows in \(S\) with exactly two \(1\)'s. \\
			We have two crucial indications about this \(S\). The first one is that it is a submatrix of \(A\). This implies that it has to fulfill the requirement that if two colours are chosen in one row, one of them must be light and the orther dark. We do not reassign colours, hence this must still hold. The second one is that we have two colours in each row of \(S\). \\
			Hence \(S\) would represent a legitimate incidence matrix according to the definition of (f). \\
			Yet we came to prove in (f)(i), that all incidence matrices as described in (f) do have linearly dependent columns. In the case of square matrices, this implies that they are not 'full-rank' or 'invertible'. This again implies that the determinant must be 0. We can conclude \(det|S|=0 \)/
	\end{enumerate} 
	The case distinction is exhaustive and leads a desired result for each case. Hence we have proven the desired statement.
	
	%%%%% iii)
	\item 
	In order to prove that every basic feasible solution is integral, we first look at the definition of a basic feasible solution. Any basic feasbile solution must satisfy n linearly independent constraints with equality. We say that a basic feasible solution \(\tilde{x}\) solves \(\tilde{A} \dot x = \tilde{b}\) where \(\tilde{A}\) and \(\tilde{b}\) represent n linearly independent constraints from either \(Ax \geq 1_m\) or \(x \geq 0 \). \\
	Cramer's rule tells us that the \(j\)-th component of our basic feasible solution can be computed by:
	\[ \tilde{x_j} = \frac{det|\tilde{A_j}|}{det|\tilde{A}|},\ \ \ \ \ \ \  j \in \{1,..,n\}\]
	where \(\tilde{A_j}\) corresponds to \(\tilde{A}\) and replacing its jth column with \(\tilde{b}\).
	In order to argue that all \(\tilde{x_j}\) are integers, we will look at both numerator and denominator of the fraction. 
	\begin{enumerate}
		\item \(det|\tilde{A_j}|\) \\
			\(\tilde{A}\) contains only 0's and 1's. So does \(\tilde{b}\). Hence \(\tilde{A_j}\) only contains 0's and 1's. When calculating the determinant we will only add, multiply and adapt the sign of elements. Adding, multiplying and changing signs of integers will result in an integer. Therefore \(det|\tilde{A_j}| \in \mathbb{N}\).
		\item \({det|\tilde{A}|}\) \\
			We know that the rows originating from the constraint \(x \geq 0\) containt exaclty one \(1\). For all those rows, we recursively compute minors, which always consist of the determinant of a minor multiplied by either \(1\) or \(-1\). When we run out of those rows, we only have rows left from the constraint \(Ax \geq 1_m\). Thusly, those remaining rows stem from A, in other words, the matrix of remaining rows is a submatrix of A. as we have shown in (ii), this submatrix yields a determinant of either -1, 0 or 1. Therefore  \({det|\tilde{A}|}\) is of the form \( (-1)^k\cdot det|M|,\ det|M\ \in \{-1,0,1\}\).\\
			As \(\tilde{A}\) contains n linearly independent rows, we know that its determinant cannot equal \(0\). \\
			It follows that \({det|\tilde{A}|} \in \{-1,1\}\).
	\end{enumerate}
	We can come to the conclusion that \( \forall j \in \{1,..,n\},\ \tilde{x_j} = det|\tilde{A_j}|\ or\ \tilde{x_j} = -det|\tilde{A_j}|\). As \(det|\tilde{A_j}| \in \mathbb{N},\ x_j \in \mathbb{N}\). Hence \(x\) is integral.
	
	%%%%% iv)
	\item
		We know from (b) that \(\tau^*(F)\) is the optimal value of the feasible and bounded linear program LP-T. \\
		Theorem 6.1 allows us to conlcude that a basic feasible solution can attain this optimal value \(\tau^*(F)\).
		We have shown in iii), that all basic feasible solutions are integral. \\
		Combining this knowledge, we know that there is an integral optimal basic feasible solution. \\
		We have argued in b) that \(\tau(F)\) is the optimal value of an integer linear program, which is only more restrive than LP-T. Precisely, it only allows components of the solution to be either 0 or 1. As we already know that we can reach the optimal value of LP-T with integral value, it only remains to show that the optimal basic solution does not contain integers other than 0 and 1, to prove that both optimal values must be equal. \\
		We will prove this via contradiction. \\
		Let's assume that the optimal basic feasible solution \(x\) of LP-T has a component \(j'\) of value \(k \notin \{0,1\}\).	As \(x \geq 0\) holds for all feasible solutions, \(k\) must be positive. To be explicit, \(k \in \{2,3,..\}\). Let's look at the constraints \(Ax \geq 1\). For each row \(i\) in \(A\) in which the \(j'\)th component is 0, the value of \(x_j'\) does not matter. For all others, the constraint corresponds to the following: 
		\[ A_ix \geq 1 \Leftrightarrow \sum_{j=1}^m A[i][j]*x[j] = 1 \cdot k + \sum_{j=1, j \neq j'}^m A[i][j]*x[j] \geq 1\]
		We notice that all those constraints would also be satisfied with \(k=1\). As we know that all constraints would still be satisfied for \(k=1\), no other component would have to be adapted and the cost function is simply adding up the components of \(x\), our \(x\) was not optimal. \\
		We now know that we can add the constraint \(x \in \{0,1\}^n\) to LP-T without influencing the optimal value \(\tau^*(F)\). Yet this corresponds exactly to the integer Linear Program with optimal value \(\tau(F)\). Hence \(\tau^*(F) = \tau(F)\). 
	%%%%% v)
	\item We will prove \( \mu(F) = \tau(F)\) by showing that \( \mu^*(F) = \mu(F)\). This suffices as we know from (d) that \( \mu^*(F) = \tau^*(F) \) as well as \(\tau^*(F)  = \tau(F)\) from (f)(iv). \\
	
	\end{enumerate}
	
	
\end{enumerate}
\end{homeworkProblem}



%----------------------------------------------------------------------------------------
\end{document}
PreferencesEnglish
