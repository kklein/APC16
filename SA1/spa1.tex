%%%%%%%%%%%%%%%%%%%%%%%%%%%%%%%%%%%%%%%%%
% Programming/Coding Assignment
% LaTeX Template
%
% Original author:
% Ted Pavlic (http://www.tedpavlic.com)
%
%
% This template uses a Perl script as an example snippet of code, most other
% languages are also usable. Configure them in the "CODE INCLUSION 
% CONFIGURATION" section.
%
%%%%%%%%%%%%%%%%%%%%%%%%%%%%%%%%%%%%%%%%%

%----------------------------------------------------------------------------------------
%	PACKAGES AND OTHER DOCUMENT CONFIGURATIONS
%----------------------------------------------------------------------------------------

\documentclass{article}
\usepackage[utf8]{inputenc}

\usepackage[german]{babel}
\usepackage{amsmath}
\usepackage{amsfonts}
\usepackage{fancyhdr} % Required for custom headers
\usepackage{lastpage} % Required to determine the last page for the footer
\usepackage{extramarks} % Required for headers and footers
\usepackage[usenames,dvipsnames]{color} % Required for custom colors
\usepackage{graphicx} % Required to insert images
\usepackage{listings} % Required for insertion of code
\usepackage{courier} % Required for the courier font
\usepackage{enumerate} % used for enumerate args
\usepackage{multicol} % columns

\usepackage{pgf} 
\usepackage{tikz}
%\usepackage{forest} % treees :D
%\usetikzlibrary{arrows,automata} %for FSM

% Custom commands
\DeclareMathOperator{\Kl}{Kl} %Klassen von Zuständen

\usepackage{mathtools}
\DeclarePairedDelimiter{\ceil}{\lceil}{\rceil}
% Shamelessly copied from http://tex.stackexchange.com/questions/43008/absolute-value-symbols
\DeclarePairedDelimiter\abs{\lvert}{\rvert} % nice |x|
\DeclarePairedDelimiter\norm{\lVert}{\rVert} % nice ||x||
% Swap the definition of \abs* and \norm*, so that \abs
% and \norm resizes the size of the brackets, and the 
% starred version does not.
\makeatletter
\let\oldabs\abs
\def\abs{\@ifstar{\oldabs}{\oldabs*}}
\let\oldnorm\norm
\def\norm{\@ifstar{\oldnorm}{\oldnorm*}}
\makeatother


% Margins
\topmargin=-0.45in
\evensidemargin=0in
\oddsidemargin=0in
\textwidth=6.5in
\textheight=9.0in
\headsep=0.25in

\linespread{1.1} % Line spacing

% Set up the header and footer
\pagestyle{fancy}
\lhead{\hmwkAuthorName} % Top left header
%\chead{\hmwkClass\ (\hmwkClassInstructor\): \hmwkTitle} % Top center head
%\rhead{\firstxmark} % Top right header
\rhead{}
\lfoot{\lastxmark} % Bottom left footer
\cfoot{} % Bottom center footer
\rfoot{Seite\ \thepage\ von\ \protect\pageref{LastPage}} % Bottom right footer
\renewcommand\headrulewidth{0.4pt} % Size of the header rule
\renewcommand\footrulewidth{0.4pt} % Size of the footer rule

\setlength\parindent{0pt} % Removes all indentation from paragraphs

%----------------------------------------------------------------------------------------
%	CODE INCLUSION CONFIGURATION
%----------------------------------------------------------------------------------------

%\definecolor{MyDarkGreen}{rgb}{0.0,0.4,0.0} % This is the color used for comments
%\lstloadlanguages{Pascal} % Load Pascal syntax for listings, for a list of other languages supported see: ftp://ftp.tex.ac.uk/tex-archive/macros/latex/%contrib/listings/listings.pdf
%\lstset{language=Perl, % Use Pascal in this example
%        frame=single, % Single frame around code
%        basicstyle=\small\ttfamily, % Use small true type font
%        keywordstyle=[1]\color{Blue}\bf, % Pascal functions bold and blue
%        keywordstyle=[2]\color{Purple}, % Pascal function arguments purple
%        keywordstyle=[3]\color{Blue}\underbar, % Custom functions underlined and blue
%        identifierstyle=, % Nothing special about identifiers                                         
%        commentstyle=\usefont{T1}{pcr}{m}{sl}\color{MyDarkGreen}\small, % Comments small dark green courier font
%        stringstyle=\color{Purple}, % Strings are purple
%        showstringspaces=false, % Don't put marks in string spaces
%        tabsize=5, % 5 spaces per tab
%        %
%        % Put standard Pascal functions not included in the default language here
%        morekeywords={rand},
%        %
%        % Put Pascal function parameters here
%        morekeywords=[2]{on, off, interp},
%        %
%        % Put user defined functions here
%        morekeywords=[3]{test},
%        %
%        morecomment=[l][\color{Blue}]{...}, % Line continuation (...) like blue comment
%        numbers=left, % Line numbers on left
%        firstnumber=1, % Line numbers start with line 1
%        numberstyle=\tiny\color{Blue}, % Line numbers are blue and small
%        stepnumber=5 % Line numbers go in steps of 5
%}

\definecolor{dkgreen}{rgb}{0,0.6,0}
\definecolor{gray}{rgb}{0.5,0.5,0.5}
\definecolor{mauve}{rgb}{0.58,0,0.82}

\lstset{frame=tb,
  language=Java,
  aboveskip=3mm,
  belowskip=3mm,
  showstringspaces=false,
  columns=flexible,
  basicstyle={\small\ttfamily},
  numbers=none,
  numberstyle=\tiny\color{gray},
  keywordstyle=\color{blue},
  commentstyle=\color{dkgreen},
  stringstyle=\color{mauve},
  breaklines=true,
  breakatwhitespace=true,
  tabsize=3
}

% Creates a new command to include a perl script, the first parameter is the filename of the script (without .p), the second parameter is the caption
\newcommand{\pascalscript}[2]{
\begin{itemize}
\item[]\lstinputlisting[caption=#2,label=#1]{#1.p}
\end{itemize}
}

%----------------------------------------------------------------------------------------
%	DOCUMENT STRUCTURE COMMANDS
%	Skip this unless you know what you're doing
%----------------------------------------------------------------------------------------

% Header and footer for when a page split occurs within a problem environment
%\newcommand{\enterProblemHeader}[1]{
%\nobreak\extramarks{#1}{#1 continued on next page\ldots}\nobreak
%\nobreak\extramarks{#1 (continued)}{#1 continued on next page\ldots}\nobreak
%}

% Header and footer for when a page split occurs between problem environments
%\newcommand{\exitProblemHeader}[1]{
%\nobreak\extramarks{#1 (continued)}{#1 continued on next page\ldots}\nobreak
%\nobreak\extramarks{#1}{}\nobreak
%}

\setcounter{secnumdepth}{0} % Removes default section numbers
\newcounter{homeworkProblemCounter} % Creates a counter to keep track of the number of problems

\newcommand{\homeworkProblemName}{}
\newenvironment{homeworkProblem}[1][Exercise \arabic{homeworkProblemCounter}]{ % Makes a new environment called homeworkProblem which takes 1 argument (custom name) but the default is "Problem #"
\stepcounter{homeworkProblemCounter} % Increase counter for number of problems
\renewcommand{\homeworkProblemName}{#1} % Assign \homeworkProblemName the name of the problem
\section{\homeworkProblemName} % Make a section in the document with the custom problem count
%\enterProblemHeader{\homeworkProblemName} % Header and footer within the environment
}{
%\exitProblemHeader{\homeworkProblemName} % Header and footer after the environment
}

\newcommand{\problemAnswer}[1]{ % Defines the problem answer command with the content as the only argument
\noindent\framebox[\columnwidth][c]{\begin{minipage}{0.98\columnwidth}#1\end{minipage}} % Makes the box around the problem answer and puts the content inside
}

\newcommand{\homeworkSectionName}{}
\newenvironment{homeworkSection}[1]{ % New environment for sections within homework problems, takes 1 argument - the name of the section
\renewcommand{\homeworkSectionName}{#1} % Assign \homeworkSectionName to the name of the section from the environment argument
\subsection{\homeworkSectionName} % Make a subsection with the custom name of the subsection
%\enterProblemHeader{\homeworkProblemName\ [\homeworkSectionName]} % Header and footer within the environment
}{
%\enterProblemHeader{\homeworkProblemName} % Header and footer after the environment
}

%----------------------------------------------------------------------------------------
%	NAME AND CLASS SECTION
%----------------------------------------------------------------------------------------

\newcommand{\hmwkTitle}{Special Assignment 1} % Assignment title
\newcommand{\hmwkDueDate}{20\ October\ 2015} % Due date
\newcommand{\hmwkClass}{Algorithms, Probability and Computing} % Course/class
\newcommand{\hmwkClassInstructor}{Prof. Steger, Prof. Holenstein, Prof. Welzl} % Teacher/lecturer
\newcommand{\hmwkAuthorName}{Kevin Klein} % Your name

%----------------------------------------------------------------------------------------
%	TITLE PAGE
%----------------------------------------------------------------------------------------

\title{
\vspace{2in}
\textmd{\textbf{\hmwkClass:\ \hmwkTitle}}\\
\normalsize\vspace{0.1in}\small{Due\ \hmwkDueDate}\\
\vspace{0.1in}\large{\textit{\hmwkClassInstructor}
\vspace{3in}
}}
\author{\textbf{\hmwkAuthorName}}
\date{} % Insert date here if you want it to appear below your name

%----------------------------------------------------------------------------------------

\begin{document}

\maketitle

%----------------------------------------------------------------------------------------
%	TABLE OF CONTENTS
%----------------------------------------------------------------------------------------

%\setcounter{tocdepth}{1} % Uncomment this line if you don't want subsections listed in the ToC

\addtocounter{homeworkProblemCounter}{0}
\newpage
%\tableofcontents
%\newpage

%----------------------------------------------------------------------------------------
%	PROBLEM 1
%----------------------------------------------------------------------------------------

\begin{homeworkProblem}
Let $P$ be the input set of points with $|P| = n$, with general position.

\begin{enumerate}[(1)]
\item - \(\mathcal{O}(n \cdot log(n)) \) \\
	Compute the Convex Hull of the set $P$ and add all $. \in P$ to the queue $Q$. Runtime boundaries and correctnes can be guaranteed by using, for instance, Graham's algorithm,
\item - \(\mathcal{O}(n \cdot log(n)) \) \\
	Compute the Voronoi diagram of the set $P$. Runtime boundaries and correctnes can be guaranteed by using, for instance, Fortune's algorithm.
\item - \(\mathcal{O}(n \cdot log(n)) \) \\
	Determine all intersections of the edges of the Voronoi diagram with the convex hull. Add every intersection to the queue $Q$, while saving a reference from the intersection point to the neighbouring points from the initial graph. A single intersection can be determined within \(\mathcal{O}(log(n))\) time. [1] There are only \(\mathcal{O}(n)\) edges in the Voronoi diagram [2], therefore we query intersections \(\mathcal{O}(n)\) many times. This yields an overall runtime of \(\mathcal{O}(n \cdot log(n)) \).
\item - \(\mathcal{O}(n) \) \\
	Iterate over all items in the queue. For each, check what the minimal distance to its closest point is. Save the maximum of the current value, initialized with 0 and the minimal distance to a neighbouring point of P. 
\item - \(\mathcal{O}(1) \) \\
	Return the point associated with the maximum value.
\end{enumerate}

\emph{Soundness} \\
If an point $p$ is returned by the algorithm, it is either part of $P$ or an intersection of a Voronoi diagram edge and the convex hull, i.e. a point lying on the interior of a segment of the convex hull.
\begin{enumerate} [a)]
\item $p \in P$ \\
	By definition, we know the three closest points of $P$ from $p$ as well as the minimum of the distances to them. Therefore, the radius can simply be defined as this distance, which assures that no $p' \in P$ is also $p' \in Int(C)$. As the only $p \in P$ populating $Q$ come from the convex Hull, we assure that $c \in conv(P)$ by defining $c = p$. 
\item $p \notin P$
	
\end{enumerate}
\emph{Completeness}

[1] APC script, Chapter 2.1 Point/Line Relative to a Convex Polygon, paragraph 'A Line Hiiting a Convex Polygon', p. 40  
[2] APC script, Chapter 2.3 Planar Point Location - More exmaples, paragrahp 'Closest Point in the Plane - the Post offce Problem', p.55
\end{homeworkProblem}
%----------------------------------------------------------------------------------------
%	PROBLEM 2
%----------------------------------------------------------------------------------------

\begin{homeworkProblem}
\end{homeworkProblem}

%----------------------------------------------------------------------------------------
%	PROBLEM 3
%----------------------------------------------------------------------------------------

\begin{homeworkProblem}
\begin{enumerate}[(a)]
	\item 			% (a)
		\( a_0 = 7, a_1 = 1 + 2 * a_0 = 15 \\
		\forall n \geq 2: \)
		\begin{align}
		 a_n &= 1 + 2 * \sum_{i=1}^{n} a_{i-1} \\
		 a_{n-1} &= 1 + 2 * \sum_{i=1}^{n} a_{i-1} \\
		\Rightarrow a_n - a_{n-1} &= 1 + 2 * \sum_{i=1}^{n} a_{i-1} - ( 1 + 2 * \sum_{i=1}^{n} a_{i-1} ) = 2 * a_{n-1}\\
		\Rightarrow a_n & = 3 * a_{n-1} \\
		&= \dots \\
 		&= 3 ^ {n-1} * a_1 \\
		&= 3^{n-1} * 15 \\
		\end{align}
	\item 			% (b)
		\( L_n := length\ of\ a\ central\ path\ in\ a\ BST\ of\ size\ n\ nodes\ \\
		l_n := \mathbb{E}(L_n) \\
		l_0 = 0, l_1 = 1, l_2 = 1/2 \cdot 2 + 1/2 \cdot 1 = 3/2\\
		\forall n \geq 3:\)
		\begin {align}
			l_n &= \sum_{i=1}^{n} ( \mathbb{E}[L_n | root = i] \cdot \Pr[root=i] ) \\
			&=  \sum_{i=1}^{n} ( \mathbb{E}[L_n | root = i] \cdot 1/n ) \\
  			&=  1/n \cdot \sum_{i=1}^{n} ( \mathbb{E}[L_n | root = i]) \\
    			&=  1/n \cdot \sum_{i=1}^{n} ( \mathbb{E}[L_{i-1}] +1) \\
 			&=  1 + 1/n \cdot \sum_{i=1}^{n} (l_{i-1}) \\
		\end{align}
		Furthermore, \( l_{n-1} = 1/n \cdot \sum_{i=1}^{n-1} (l_{i-1}) \)
 		\begin {align}
		  	n \cdot l_n - (n-1) \cdot  l_{n-1} &= n +  \sum_{i=1}^{n} (l_{i-1}) - ((n-1) +  \sum_{i=1}^{n-1} (l_{i-1}) \\
			&= 1 + l_{n-1} \\
			\Rightarrow n \cdot l_n &= 1 + n \cdot l_{n-1} \\
			\Rightarrow l_n &= 1/n +  l_{n-1} \\
			&= 1/n +  1/(n-1) + \dots + l_3 + l_2 \\
 			&= 1/n +  1/(n-1) + \dots + l_3 + 3/2 \\
  			&= 1/n +  1/(n-1) + \dots + l_3 + 1/2 + 1 \\
  			&= H_n \\
 		\end{align}
		We see that this also holds for \(n=2\) and \(n=1\) and therefore: \( \forall n \in \mathbb{N} - \{0\}. \)
	\item  			% (c)
		\( P_0^{(1)} = P_1^{(1)} = P_2^{(1)} = 0, P_3^{(1)} = 1/3, P_4^{(1)} = 1/4 \cdot 1/3 = 1/12 \\ 	
		\forall n \geq 5: \)
		\begin{align}
			P_n^{(1)} &= \sum_{i=1}^n (( P_n^{(1)} | root=i) \cdot \Pr[root=i]) \\
			&= \sum_{i=1}^n (( P_n^{(1)} | root=i) \cdot 1/n \\
			&= 1/n \cdot \sum_{i=1}^n (( P_n^{(1)} | root=i)   \\
			&= 1/n \cdot \sum_{i=1}^n P_{i-1}^{(1)}   \\
		\end{align}
		Furthermore, \( P_{n-1}^{(1)} =  1/(n-1) \cdot \sum_{i=1}^{n-1} (( P_{i-1}^{(1)} | root=i) \)
		\begin{align}
			\Rightarrow n \cdot P_n^{(1)} - (n-1) \cdot P_{n-1}^{(1)} &= \sum_{i=1}^n P_{i-1}^{(1)} - ( \sum_{i=1}^{n-1} P_{i-1}^{(1)} ) \\
			&= P_{n-1}^{(1)} \\
			\Rightarrow n \cdot P_n^{(1)} &= n \cdot P_{n-1}^{(1)} \\
			\Rightarrow P_n^{(1)} &=  P_{n-1}^{(1)} \\
			&=  P_{n-2}^{(1)} = \dots =  P_{4}^{(1)} \\
			&= 1/12
		\end{align}
		We see that this also holds for \(n=4\) and therefore: \( \forall n \in \mathbb{N}, n \geq 4. \)
	\item 			% (d)
\end{enumerate}
\end{homeworkProblem}

%----------------------------------------------------------------------------------------
%	PROBLEM 4
%----------------------------------------------------------------------------------------

\begin{homeworkProblem}
	
\end{homeworkProblem}



%----------------------------------------------------------------------------------------
\end{document}
PreferencesEnglish
